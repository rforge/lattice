\documentclass[10pt]{article}

\newcommand{\pkg}[1]{{\bfseries #1}}
\newcommand{\code}[1]{{\ttfamily #1}}

\title{A Naming Scheme for Lattice Grobs} 
\author{Paul Murrell}
\date{}

\usepackage[text={6.5in,8.5in},centering]{geometry}
\usepackage[round]{natbib}
\usepackage{alltt}
\usepackage{graphicx}
\usepackage{url}
\usepackage{hyperref}

% \usepackage{Sweave}
% \setkeys{Gin}{width=0.98\textwidth}

\begin{document}
\maketitle

\raggedright


\section*{A Naming Scheme for \pkg{lattice} Grobs}

% Lots of really long grob names
\sloppy 

All grobs produced by \pkg{lattice} should be provided with names
using the following guidelines:

\begin{itemize}
\item
All names should be generated using a call to \code{trellis.grobname()}.

The argument \code{name} provides the main descriptive name for the grob.

The argument \code{prefix} is used to distinguish between 
different plots on the same page (this defaults using
\code{lattice.getStatus()}).  This prefix is prepended to all grob names.

The argument \code{type} is used to distinguish between grobs that are
once-per-plot and grobs that are once-per-panel.  The latter are
produced by specifying \code{"panel"}, \code{"strip"}, or
\code{"strip.left"} for \code{type}.  In those cases, a suffix is
added to the grob name of the form \code{"panel.i.j"}, where the
\code{i} an \code{j} come from the arguments \code{row} and
\code{column} and those arguments
 get default values from \code{lattice.getStatus()}).
The naming of strip grobs is similar, except that it can be more
complicated when there are multiple conditioning variables (and hence
multile strips per panel).  In that case, the grob suffix will be of
the form \code{"given.k.strip.i.j"}, where \code{k} comes from
the argument \code{which.given}, but that only occurs if
the argument \code{which.panel} has length greater than 1.
Both of those arguments get their default
values from \code{lattice.getStatus()} so there is usually no need
to explicitly provide them.

The argument \code{group} is used to distinguish between grobs that are
once-per-group (within a panel).  A value greater than zero causes
a suffix of the form \code{"group.i"} to be appended to the grob name
(before any panel suffix).

The argument \code{type} can also be \code{"key"} or \code{"colorkey"}.
These are used to name the components of \code{draw.key()} and
\code{draw.colorkey()}.  The latter is simple because all components
are distinct, so you get names of the form \code{"plot\_01.colorkey.border"}.
The \code{draw.key()} components are a little more complex because
the contents of a legend can be quite flexible.  There are some
straightforward components like \code{"plot\_01.key.title"},
but the general contents are of the form \code{"plot\_01.key.text.1.1"}
where the \code{"1.1"} refers to the appropriate column and row 
of the legend contents.


\item
In some cases, particularly for once-per-plot grobs, there are one-off
\pkg{grid} function calls.  A good example is the background rect for
the entire plot that is drawn by \code{plot.trellis()}.  Such cases 
only require a descriptive name and \code{type=""}.  For example ...

\begin{verbatim}
grid.rect(name=trellis.grobname("background", type=""),
          gp = gpar(fill = bg, col = "transparent"))
\end{verbatim}

... produces something like \code{"plot\_01.background"}.

There are other examples of this sort of once-per-plot grob in
some panel functions.

\item
Most drawing of grobs occurs through a call to one of the primitive
functions, \code{ltext()}, \code{llines()}, etc (or their ``skins'', 
\code{panel.text()}, \code{panel.lines()}, etc).  The 

These functions produce a basic descriptive name, e.g., \code{"text"}
for \code{ltext()} and, by default, add panel information.  
These functions also add
group information if appropriate (this is based on detecting a 
special argument that \code{panel.superpose()} passes down, i.e., 
a test that the function has been called by \code{panel.superpose()}).
For example, a direct call to \code{panel.points()} will produce a grob name
of the form \code{"plot\_01.points.panel.1.1"}.

These functions also 
have an \code{identifier} argument which, if non-\code{NULL},
gets prepended to the descriptive grob name.  This allows higher-level
panel functions to provide a distinctive prefix for primitive grobs.
For example, a call to \code{panel.xyplot()}, which calls 
\code{panel.points()}, will produce a grob name
of the form \code{"plot\_01.xyplot.points.panel.1.1"}.

The \code{identifier} argument also allows the user to produce
distinct names for primitive grobs when writing a panel function.
For example, ...

\begin{verbatim}
mypanel <- function(x, y, ...) {
    panel.xyplot(x, y, ...)
    panel.points(a, b, identifier = "extra")
}
\end{verbatim}

... will produce something like \code{"plot\_01.xyplot.points.panel.1.1"}
for the default points and \code{"plot\_01.extra.points.panel.1.1"} for
the custom points.

There is also a \code{name.type} argument, which is a character value
indicating whether to add panel (or strip) information to the grob name.
This corresponds directly to the \code{type} argument to 
\code{trellis.grobname()}.
The additional information can be left out by specifying \code{name.type=""}.

\item
Higher-level panel functions come in two types.  Some, 
like \code{panel.abline()}, make direct \pkg{grid} calls, so they 
have a similar set up to the primitive panel functions above.
Others make calls to the primitive panel functions, in which case they
provide descriptive \code{identifier} arguments for those calls.

For example, \code{panel.densityplot()} may call \code{panel.points()} to
draw raw values and \code{panel.lines()} to draw the density curve.
For both calls, it specifies
\code{identifier = "density"}, so that the resulting grobs have names
of the form \code{"plot\_01.density.points.panel.1.1"} and
\code{"plot\_01.density.lines.panel.1.1"}.

These higher-level panel functions also have an \code{identifier} argument
so that users can call them in custom panel functions and provide
a unique identifier for grobs that they add.

\item 
Any user customisations to a \pkg{lattice} plot 
that use raw \pkg{grid} calls can specify whatever
\code{name} for a grob that they like.

\end{itemize}

\end{document}
